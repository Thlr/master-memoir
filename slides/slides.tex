\documentclass[12pt, aspectratio=43]{beamer}
\usepackage[backend=bibtex]{biblatex} % bibliography

% BEAMER TEMPLATE AND THEME
\setbeamertemplate{footline}[frame number]
\usetheme{Estonia}
\mode<presentation>

% TITLE PAGE
\title{Kubernetes cluster simulator based on Batsim}
\title{Development and evaluation of a Kubernetes cluster simulator based on
Batsim}

\author{\textbf{Presented by:} Théo Larue\\\textbf{Supervised by:} Olivier
Richard \& Michael Mercier}

\date{August 31, 2020}

\institute[Théo LARUE]{Université Grenoble Alpes}

\titlegraphic{
	\includegraphics[height=5ex]{../imgs/uga-logo.png}\hspace{2ex}
	\includegraphics[height=6ex]{../imgs/ENSIMAG.png}\hspace{2ex}
	\includegraphics[height=5ex]{../imgs/Logo-LIG.jpg}\hspace{2ex}
	\includegraphics[height=5ex]{../imgs/ryax-logo.png}
}

\begin{document}

\frame{\titlepage}

\begin{frame}\frametitle{Table of contents}\tableofcontents
\end{frame}

\section{Introduction}
\begin{frame}{Computer infrastructures}
	Distributed systems, many domains: Grid, Edge, HPC, Cloud, P2P, Volunteer, Cluster. 

	These systems increase in complexity. (-> some ref to illustrate the
	size of supercomputers)
\end{frame}

\begin{frame}{Studying distributed systems}
	Why studying these infra? To test a system performances under varying
	loads, applications, scheduling policies, system size and topology. Or
	to develop new RJMS or research new scheduling algorithms.
\end{frame}

\begin{frame}{Studying distributed systems}
	How to study these infra? Too many elements and interactions to
	consider, so no theoretical study. 

	Real experiments are too costly (both in time and resources) and not
	reproducible.
\end{frame}

\begin{frame}{Studying distributed systems}
	First solution: emulation resolves the issue of reproducibility.

	Second option: simulation resolves both reproducibility and scalability
	issues + refs on existing simulators.
\end{frame}

\section{Literature review}
\begin{frame}{Domain specific simulators}
	refs on domain specific simulators.
\end{frame}

\begin{frame}{Software specific simulators}
	YARNSim, SLURM simulator
\end{frame}

\begin{frame}{Publication specific simulators}
	``Publish and perish'' - Milian Poquet
\end{frame}

\begin{frame}{Batsim}
\end{frame}

\begin{frame}{Batsim - related work}
\end{frame}

\begin{frame}{Kubernetes}
\end{frame}

\begin{frame}{Kubernetes cluster simulation}
\end{frame}

\section{Integrating Kubernetes schedulers to Batsim}
\begin{frame}{Batsim concepts}
\end{frame}

\begin{frame}{Kubernetes concepts}
\end{frame}

\begin{frame}{Batkube integration with Kubernetes}
\end{frame}

\begin{frame}{Time interception}
\end{frame}

\begin{frame}{Time synchronization}
\end{frame}

\section{Study of the simulator}
\begin{frame}{Studied workloads and platforms}
\end{frame}

\begin{frame}{Minimum delay}
	TODO for future work, study min delay effect on makespan and mwt
\end{frame}

\begin{frame}{Timeout}
\end{frame}

\begin{frame}{Maximum simulation timestep}
\end{frame}

\begin{frame}{Experimentation on a real cluster}
\end{frame}

\begin{frame}{Deviation with reality}
\end{frame}

\section{Discussion and future work}
\begin{frame}{Capabilities of Batkube}
\end{frame}

\begin{frame}{Features to implement}
\end{frame}

\begin{frame}{Limitations}
\end{frame}

\begin{frame}{Perspectives for future work}
\end{frame}

\begin{frame}[allowframebreaks]
        \frametitle{References}
        %\bibliographystyle{amsalpha}
	\bibliography{../report/biblio.bib}
\end{frame}

\end{document}


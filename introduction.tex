\chapter{Introduction}

TODO: Make another pass on that section after everything else is redacted (or
at least the soa)\\

The need for scalable computing infrastructure has increased tremendously in
the last decades. Nearly every field of computer science, from research to the
service industry, now needs a proper infrastructure and by 2025, computation
technology could reach a fourth of the global electricity
spending\cite{andrae2017total}.  Even the public sector is now in need for
efficient distributed infrastructure as the concept of smart cities is
developing.

Organizations generally know what type of infrastructure will meet their needs.
It can take the form of Big Data centers to store and analyze data,
High-Performance Computers for computing intensive tasks or GPU banks for
machine learning or crypto-currency mining.  However, studying those
infrastructures extensively is much more challenging.  As these computers reach
scales in the order of warehouses\cite{barroso2018datacenter}, quantifying a
system's performance under varying loads, applications, scheduling policies and
system size quickly becomes undoable without expensive real world experiments.
In fact, the nature of scheduling problems\cite{scheduler-complexity} alone
make theoretical studies hard.  This is an issue for organizations as they rely
on thoses studies to determine the size of the required system or choose
optimal scheduling
policies.\\

Simulation allows to tackle these issues by enabling users to draw conclusions
empirically without the need to fire up real workloads. Indeed, running an
entire experimental campaign on a real system represents consequencial costs
both in time and money. With simulation, The gain in both time and spent energy
can be extreme : a HPC job spanning months on a real system can be resolved in
a matter of minutes on any domestic computer.  Another major point is that it
also brings reproducibility to these experiments, that otherwise would have to
be run on the exact same systems as their first iteration. With simulation, one
can recreate the same conditions for any experiment anywhere they want, and
expect the same results.\\

However, simulations need to be run with sound models for the results to be
exploitable and in that regard, simulators usually fall under several
pitfalls\cite{poquet:tel-01757245}. Very often simulators are implemented at
the same time as new schedulers or \textit{Resource and Jobs Management
Systems (RJMS)}\footnote{The RJMS is the software at the core of the cluster. It is a
synonym for a scheduler and manages resources, energy consumption, users' jobs
life-cycle and implements scheduling policies.} in order to validate their
algorithms. Thus, they are strongly coupled together and are not usable with
any other software. They are either shipped with the software itself or worst,
they are never released and discarded at the end of the development process.
Moreover, still according to \cite{poquet:tel-01757245}, strong coupling may
lead to unrealistic models. In that case cluster resources can be accessed with
ease by the scheduler, resulting in it having very precise information about the
system state to take its decisions.  This conflicts with the real world as a
scheduler may not have access to all the information it wants, or may suffer
from latency when getting it from the system.\\

To try and assess these issues a team of researchers at the LIG developed
Batsim\cite{dutot:hal-01333471} which is a general purpose infrastructure
simulator with modularity and separation of concerns in mind. Batsim is based
on SimGrid\cite{casanova:hal-01017319} which is a framework for developing
simulators for distributed computer systems. Simgrid is now a 20 years old
framework that has been used in many
projects\footnote{https://simgrid.org/usages.html}, making it a sound choice to
run scalable and accurate models of the reality.

Batsim was designed to support algorithms written in any languages, as long as
they support its communication protocol. It means that, while any scheduler
found in the wild can potentially be run on a Batsim simulation, they still
have to be adapted to make them compatible. This master's project is dedicated
on developing an interface between Batsim and
Kubernetes\footnote{https://github.com/kubernetes/kubernetes/} schedulers in
order to run Kubernetes clusters simulations. Kube\footnote{Another term to
designate Kubernetes. It is also sometimes called k8s.} is an open source
container management software widely exploited in the industry for its ease of
use and wide range of capabilities. It has freed developers from the cumbersome
task of setting up low level software infrastructure on their servers and
automates maintenance, scaling and administration of their applications. For
all these reasons it has become a de-facto solution for any organization that
wishes to build new internet platforms from the ground up.\\

TODO : what we where able to do (summary of the simulator capabilities,
experimentations, results)

